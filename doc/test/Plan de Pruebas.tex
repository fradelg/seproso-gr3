%%%%%%%%%%%%%%%%%%%%%%%%%%%%%%%%%%%%%%%%%%%%%%%%%%%%%%%%%%%%%%%%%%%%%%%%%%% 
% Plan de iteracion
% 17/11/08 -> 07/12/2009
% Universidad de Valladolid
%%%%%%%%%%%%%%%%%%%%%%%%%%%%%%%%%%%%%%%%%%%%%%%%%%%%%%%%%%%%%%%%%%%%%%%%%%% 

\documentclass[a4paper,oneside,11pt]{book}

% Incluimos todos los paquetes necesarios 
\usepackage[latin1]{inputenc} % Caracteres con acentos. 
\usepackage{latexsym} % S�mbolos 
\usepackage{graphicx} % Inclusi�n de gr�ficos. Soporte para \figure
\usepackage[pdftex=true,colorlinks=true,plainpages=false]{hyperref} % Soporte hipertexto
\usepackage{rotating}
\usepackage{multirow}

% usamos cualquier medida de m�rgenes y controlamos los m�rgenes {izquierda}{derecha}{arriba}{abajo}
%\usepackage{anysize}
%\marginsize{4.9cm}{2.1cm}{1.5cm}{4.7cm}

% T�tulo, autor(es), fecha. 
\title{
%\begin{titlepage}
%  \begin{figure}
%    \begin{center}
%      \includegraphics[width=5cm]{images/portada.png}
%    \end{center}
%  \end{figure}
%\end{titlepage} \\
\Huge \textbf{SEPROSO.\\ Revisi�n [15 DIC].\\}}
\author{\huge \textit{Francisco Javier Delgado del Hoyo} \and 
	\huge \textit{Yuri Torres de la Sierra} \and 
	\huge \textit{Rub�n Mart�nez Garc�a} \and
	\huge \textit{Abel Lozoya de Diego}} 
\date{\Large Diciembre, 2008} 

\sloppy % suaviza las reglas de ruptura de l�neas de LaTeX
\frenchspacing % usar espaciado normal despu�s de '.'
\pagestyle{headings} % p�ginas con encabezado y pie b�sico

%%%%%%%%%%%%%%%%%%%%%%%%%%%%%%%%%%%%%%%%%%%%%%%%%%%%%%%%%%%%%%%%%%%%%%%%%%% 
% Comando: 
% \figura{nombre-fichero}{argumentos}{t�tulo}{etiqueta} 
% Resultado: inserta una figura. "La figura \ref{etiqueta} muestra..." 
% permite referenciar la figura desde el texto. 
% Argumentos: width=Xcm,height=Ycm,angle=Z 
%%%%%%%%%%%%%%%%%%%%%%%%%%%%%%%%%%%%%%%%%%%%%%%%%%%%%%%%%%%%%%%%%%%%%%%%%%% 
\newcommand{\figura}[4]{
  \begin{figure} 
  \begin{center} 
  \includegraphics[#2]{#1} 
  \caption{#3} 
  \label{#4} 
  \end{center} 
  \end{figure} 
} 

%%%%%%%%%%%%%%%%%%%%%%%%%%%%%%%%%%%%%%%%%%%%%%%%%%%%%%%%%%%%%%%%%%%%%%%%%%% 
% Entorno: 
% \begin{tabla}{t�tulo}{etiqueta} 
% ... (contenido tabla) 
% \end{tabla} 
% Resultado: inserta una tabla. 
% El contenido de la tabla se define mediante un entorno 'tabular'. 
% "La tabla~\ref{etiqueta}" permite referenciar la tabla. 
%%%%%%%%%%%%%%%%%%%%%%%%%%%%%%%%%%%%%%%%%%%%%%%%%%%%%%%%%%%%%%%%%%%%%%%%%%% 
\newenvironment{tabla}[2]{ 
  \begin{table} 
  \begin{center} 
  \caption{#1} 
  \label{#2} 
}{ 
  \end{center} 
  \end{table}
} 


\begin{document} % Inicio del documento 

\renewcommand{\contentsname}{Indice} 
\renewcommand{\partname}{Parte} 
\renewcommand{\chaptername}{Cap�tulo} 
\renewcommand{\appendixname}{Ap�ndice} 
\renewcommand{\bibname}{Bibliograf�a} 
\renewcommand{\figurename}{Figura} 
\renewcommand{\listfigurename}{Indice de figuras} 
\renewcommand{\tablename}{Tabla} 
\renewcommand{\listtablename}{Indice de tablas} 

% Parte inicial de la memoria: portada, t�tulo, pr�logo e �ndices
\frontmatter

\maketitle % T�tulo 

\chapter{Revisiones del documento} % Historial

\textbf{Historial de revisiones del documento}\\\\
\begin{tabular}{|c|c|p{8cm}|c|}
\hline
VERSI�N & FECHA & DESCRIPCI�N & AUTOR\\
\hline
1.0 & 30/11/08 & Revisi�n tras comienzo fase de construcci�n. & Grupo III\\
\hline
\end{tabular}


% inserta todos los ep�grafes hasta el nivel \paragraph en la tabla de contenidos
\setcounter{tocdepth}{3} 
% numera todos los ep�grafes hasta nivel \subparagraph en el cuerpo del documento
\setcounter{secnumdepth}{4} 

\tableofcontents % Tabla de contenido 
\newpage 

% Parte central de la memoria
\mainmatter

\chapter{Introducci�n.} 

\section{Prop�sito.}

\indent \indent Este documetno recoge los siguientes objetivos:
\begin{itemize}
	\item Identificar cuestiones de la herramienta que pueden causar fallos de funcionamiento.
	\item Listar los requisitos recomendados de alto nivel.
	\item recomendar las estrategias para llevar a cabo la prueba de la herramienta.
\end{itemize}

\section{�mbito.}

\indent \indent 

\section{Definiciones.}

\indent \indent \textit{V�ase} el glosario.

\section{Referencias.}

\begin{enumerate}
	\item Glosario, Grupo III, Universidad de Valladolid.
	\item SPMP, Documento del plan de proyecto, Grupo III, Universidad de Valladolid.
	\item Modelo de Casos de Uso, Grupo III, Universidad de Valladolid.
	\item Modelo de an�lisis, Grupo III, Universidad de Valladolid.
	\item Lista de Riesgos, Grupo III, Universidad de Valladolid.
	\item Modelo de dise�o, Grupo III, Universidad de Valladolid.
\end{enumerate}

\section{Vista general.}

\indent \indent  Este documetno recoge la normativa est�ndar de UPEDU para el plan de pruebas.

\chapter{Requisitos para las pruebas.}

\chapter{Estrategia de pruebas.}

\section{Tipos de pruebas}
\subsection{Pruebas funcionales}
\subsection{Pruebas de Interfaz de Usuario}
\subsection{Pruebas de integridad de la Base de Datos}
\subsection{Pruebas de rendimiento}
\subsection{Pruebas de carga de datos}
\subsection{Pruebas de seguridad}

\section{Herramientas}

\chapter{Recursos}
\section{Trabajadores}
\section{Sistema}

\chapter{Hitos del proyecto}

\chapter{Entregables}

\section{Modelo de pruebas}
\section{Resultados de pruebas}
\section{Informe de evaluaci�n de pruebas}
% Parte final de la memoria
\backmatter 


% Fin del documento
\end{document}
