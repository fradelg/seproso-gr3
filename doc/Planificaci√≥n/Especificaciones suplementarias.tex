%%%%%%%%%%%%%%%%%%%%%%%%%%%%%%%%%%%%%%%%%%%%%%%%%%%%%%%%%%%%%%%%%%%%%%%%%%% 
% Plan de iteracion
% 17/11/08 -> 07/12/2009
% Universidad de Valladolid
%%%%%%%%%%%%%%%%%%%%%%%%%%%%%%%%%%%%%%%%%%%%%%%%%%%%%%%%%%%%%%%%%%%%%%%%%%% 

\documentclass[a4paper,oneside,11pt]{book}

% Incluimos todos los paquetes necesarios 
\usepackage[latin1]{inputenc} % Caracteres con acentos. 
\usepackage{latexsym} % S�mbolos 
\usepackage{graphicx} % Inclusi�n de gr�ficos. Soporte para \figure
\usepackage[pdftex=true,colorlinks=true,plainpages=false]{hyperref} % Soporte hipertexto
\usepackage{rotating}

% usamos cualquier medida de m�rgenes y controlamos los m�rgenes {izquierda}{derecha}{arriba}{abajo}
%\usepackage{anysize}
%\marginsize{4.9cm}{2.1cm}{1.5cm}{4.7cm}

% T�tulo, autor(es), fecha. 
\title{
%\begin{titlepage}
%  \begin{figure}
%    \begin{center}
%      \includegraphics[width=5cm]{images/portada.png}
%    \end{center}
%  \end{figure}
%\end{titlepage} \\
\Huge \textbf{SEPROSO. \\Plan de iteracion.\\}}
\author{\huge \textit{Francisco Javier Delgado del Hoyo} \and 
	\huge \textit{Yuri Torres de la Sierra} \and 
	\huge \textit{Rub�n Mart�nez Garc�a} \and
	\huge \textit{Abel Lozoya de Diego}} 
\date{\Large Diciembre, 2008} 

\sloppy % suaviza las reglas de ruptura de l�neas de LaTeX
\frenchspacing % usar espaciado normal despu�s de '.'
\pagestyle{headings} % p�ginas con encabezado y pie b�sico

%%%%%%%%%%%%%%%%%%%%%%%%%%%%%%%%%%%%%%%%%%%%%%%%%%%%%%%%%%%%%%%%%%%%%%%%%%% 
% Comando: 
% \figura{nombre-fichero}{argumentos}{t�tulo}{etiqueta} 
% Resultado: inserta una figura. "La figura \ref{etiqueta} muestra..." 
% permite referenciar la figura desde el texto. 
% Argumentos: width=Xcm,height=Ycm,angle=Z 
%%%%%%%%%%%%%%%%%%%%%%%%%%%%%%%%%%%%%%%%%%%%%%%%%%%%%%%%%%%%%%%%%%%%%%%%%%% 
\newcommand{\figura}[4]{
  \begin{figure} 
  \begin{center} 
  \includegraphics[#2]{#1} 
  \caption{#3} 
  \label{#4} 
  \end{center} 
  \end{figure} 
} 

%%%%%%%%%%%%%%%%%%%%%%%%%%%%%%%%%%%%%%%%%%%%%%%%%%%%%%%%%%%%%%%%%%%%%%%%%%% 
% Entorno: 
% \begin{tabla}{t�tulo}{etiqueta} 
% ... (contenido tabla) 
% \end{tabla} 
% Resultado: inserta una tabla. 
% El contenido de la tabla se define mediante un entorno 'tabular'. 
% "La tabla~\ref{etiqueta}" permite referenciar la tabla. 
%%%%%%%%%%%%%%%%%%%%%%%%%%%%%%%%%%%%%%%%%%%%%%%%%%%%%%%%%%%%%%%%%%%%%%%%%%% 
\newenvironment{tabla}[2]{ 
  \begin{table} 
  \begin{center} 
  \caption{#1} 
  \label{#2} 
}{ 
  \end{center} 
  \end{table}
} 


\begin{document} % Inicio del documento 

\renewcommand{\contentsname}{Indice} 
\renewcommand{\partname}{Parte} 
\renewcommand{\chaptername}{Cap�tulo} 
\renewcommand{\appendixname}{Ap�ndice} 
\renewcommand{\bibname}{Bibliograf�a} 
\renewcommand{\figurename}{Figura} 
\renewcommand{\listfigurename}{Indice de figuras} 
\renewcommand{\tablename}{Tabla} 
\renewcommand{\listtablename}{Indice de tablas} 

% Parte inicial de la memoria: portada, t�tulo, pr�logo e �ndices
\frontmatter

\maketitle % T�tulo 

\chapter{Revisiones del documento} % Historial

\textbf{Historial de revisiones del documento}\\\\
\begin{tabular}{|c|c|p{8cm}|c|}
\hline
VERSI�N & FECHA & DESCRIPCI�N & AUTOR\\
\hline
0.1 & 7/11/08 & Recopilaci�n de Informaci�n Inicial. & Grupo III\\
\hline
0.2 & 17/11/08 & Primera versi�n del plan de proyecto. & Grupo III\\
\hline
\end{tabular}


% inserta todos los ep�grafes hasta el nivel \paragraph en la tabla de contenidos
\setcounter{tocdepth}{3} 
% numera todos los ep�grafes hasta nivel \subparagraph en el cuerpo del documento
\setcounter{secnumdepth}{4} 

\tableofcontents % Tabla de contenido 
\newpage 
\listoffigures % �ndice de figuras 
\newpage 
\listoftables % �ndice de tablas 
\newpage 

% Parte central de la memoria
\mainmatter

\chapter{Introducci�n.} 

\section{Prop�sito.}


\section{�mbito.}


\section{Definiciones.}

\indent  \indent V�ase el Glosario.

\section{Referencias.}

\section{Visi�n general.}


\chapter{Supuestos y dependencias.}

\chapter{Usabilidad.}
\section{<Usabilidad Requisito Uno>}

\chapter{Fiabilidad.}
\section{<Fiabilidad Requisito Uno>}

\chapter{Rendimiento.}
\section{<Rendimiento Requisito Uno>}

\chapter{Limitaciones de dise�o.}
\section{<Restricci�n de dise�o Uno>}

\chapter{Seguridad.}

\chapter{Interfaces.}
\section{Interfaz de usuario.}
\section{Interfaces hardware.}
\section{Interfaces software.}
\section{Interfaces de comunicaci�n.}

\chapter{Normas aplicables.}


% Parte final de la memoria
\backmatter 


% Fin del documento
\end{document}
